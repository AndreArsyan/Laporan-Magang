\chapter{Pendahuluan}
\section{Latar Belakang}
Perkembangan teknologi yang semakin maju seakan mendorong masyarakat untuk dapat mengerjakan pekerjaannya dengan menggunakan computer. Akan tetapi dikarenakan device dan system operasi yang beragam, maka dibutuhkan sistem yang mampu bekerja di berbagai platform, dan aplikasi berbasis web merupakan solusi untuk masalah tersebut.\\
\\
Aplikasi berbasis web memiliki kelebihan yaitu kemampuan untuk diakses di berbagai alat computer. Telah diketahui pada perusahaan ACNielsen Indonesia, karyawan menggunakan laptop sendiri dan tidak selalu menggunakan computer kantor untuk mengerjakan pekerjaannya, sehingga aplikasi berbasis web dinilai lebih efektif untuk digunakan dibandingkan dengan aplikasi desktop. Selain itu kelebihan yang dimiliki adalah pengguna tidak harus selalu menggunakan satu computer yang sama untuk dapat menggunakan aplikasi, sehingga pengguna dapat lebih fleksibel untuk dapat mengakses aplikasi. Real Time Monitoring dapat dicapai dengan aplikasi berbasis web, memudahkan user dalam memantau pekerjaan user lain tanpa harus bertemu secara langsung atau melalui berbagai halangan yang sebelumnya dimiliki pada system offline.\\
\\
PT ACNielsen Indonesia telah membuat sistem aplikasi berbasis web yang disebut dengan Web GCS, dimana website tersebut berisi kumpulan aplikasi dari sistem sistem perusahaan yang tadinya dikerjakan secara offline. Web GCS memerlukan pengembangan sehingga mampu memberikan layanan yang lebih reliable dan selaras dengan kebutuhan perusahaan. Pengembangannya adalah berupa transformasi tampilan menjadi lebih menarik dengan memanfaatkan framework Bootstrap, kemudian mengganti beberapa modul yang dikembangkan sendiri dengan modul-modul umum berbasis opensource agar lebih memudahkan developer lain dalam melakukan pengenmbangan selanjutnya.

\section{Ruang Lingkup Magang}
Kegiatan magang dilaksanan di PT ACNielsen yang berlokasi di Jl. Lapangan Ross no 1, Tebet, Jakarta Selatan. Pelaksanaan magang dimulai sejak tanggal 26 Maret 2014 hingga 26 Juni 2014. Penulis ditempatkan di Departemen Data Acquisition Customer Relationship (DA CR). Laporan magang ini akan membahas lebih jauh mengenai beberapa tugas dan tanggung jawab saat menjalani program magang pada PT ACNielsen Indonesia yang meliputi pengembangan Database Integrated System, transisi sistem offline menjadi sistem online, database Management dan membantu CR Process Management Team untuk menciptakan sistem yang efektif

\section{Tujuan Magang}
Tujuan dari pelaksanaan magang adalah mengembangkan website GCS

\section{Manfaat Magang}
Manfaat dari pelaksanaan magang bagi perusahaan adalah meningkatkan realibiltas dan stabilitas website GCS, menyajikan User Interface dan User Experience yang lebih baik serta mendapatkan sistem yang mudah untuk dikembangkan
